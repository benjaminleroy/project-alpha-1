% tex file for discussion
\par \indent While very much still a work in progress, our analysis thus far 
includes methods for both data processing and modeling voxel time courses. 
Prior to doing any serious analysis, we had to smooth the data spatially for 
each subject. We also generated a reasonable convolved time course with time 
shift corrections, based on event-related neurological stimuli with 
non-constant intervals. 

\par A basic but nevertheless important model to consider is linear regression. 
We implemented both simple and multiple regression models at the individual 
subject level. In addition to the convolved time course, our multiple linear 
regression models attempt to account for more of the noise in our data by 
including terms for event conditions, linear drift, and Fourier transforms. 
We then checked the assumptions for the fitted models and performed hypothesis 
testing on the resulting coefficients for each voxel. Though the linear 
regression models were designed to handle each voxel for each subject's data 
individually, we aggregated the data across the 24 subjects by taking the means 
of the t-statistics corresponding to each voxel. However, one major concern for 
hypothesis testing is the issue of multiple comparisons, which we attempted to 
address using the Benjamini-Hochberg procedure. Finally, we used k-means 
clustering to further identify areas of the brain with high neurological 
activity during the events of the BART study. 



