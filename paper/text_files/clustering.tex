% tex file for clustering 
\par Now that we have the across-subject average t-statistics for every voxel 
in the brain, we are left with a 3-d array of t-statistics that contain both 
negative and positive values. Instead of manually observing patterns in these 
images, we instead implemented a clustering algorithm to split the entire 3-d 
images into clusters based on the voxels' relative location to each other as 
well as the values of their t-statistics.

\par In order to find a proper clustering algorithm, we decided to treat this
problem like a grayscale image segmentation problem and implemented a
agglomerative hierarchical cluster using Ward's method. Agglomerative means
that the clusters are built bottom up which each observation starting as its
own cluster and pairs being moved up the hierarchy. Ward's method creates
clusters based on a minimum variance criterion that miniizes the total
within-cluster variances. An example of this implimentation for a 2d image is
seen here: 
\url{http://scikit-learn.org/stable/auto_examples/cluster/plot_lena_ward_segment
 ation.html}.

In our implimentation, we defined a structure to our data using a connectivity
graph in order to ensure that each cluster is spatially constrained. Also,
since our scenario uses a 3d image, the connectivity graph will also have to
take into account this extra dimension.


\par Ultimately, the goal of clustering is to have a better understanding of 
which parts of the brain are related to the signal based on the t-statistics 
from performing hypothesis tests on the coefficients from linear regression.
Once we obtain our clusters, we will both measure the within-cluster mean of
t-statistics as well measure the centroids of the clusters. By doing this, we hope
to see the parts of the brain that have the strongest relationship with the
signal and compare them to the results of the origin research paper.



