\par \indent Extending on our work with multiple linear regression we will explore modeling more of the noise in our data, with linear drift and discrete cosine transforms for general trends (similar to Fourier series as extra features). 
\par We will also be looking into (as long as this isn’t corrected with pre-processing from mentors), realignment of scans to correct for the time it takes to scan each voxel compared to the start of the scan.
\par One major concern for hypothesis testing and working with the estimated coefficients from linear modelling is the issue of multiple comparisons. This may be as simple as utilizing a Bonferroni correction, but we can also consider permutation tests or more sophisticated models (Benjamini-Hoffberg, etc). 
\par More quantitative and robust indicators for validating time series models should be implemented. One possibility is to simulate a null process by permuting the phases of the voxel’s time course after performing a Fourier transform, and then transforming the permuted data back to the original space. This way, the data is permuted but maintains the original correlated structure. We can then fit the ARIMA model to the permuted process and examine how much, if at all, the ARIMA process fitted to the observed data makes improvements over the null case. Generating confidence intervals for the parameter estimates, expanding the procedure across multiple voxels or subjects, and forecasting future observations may also be of interest. 
