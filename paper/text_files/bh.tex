% tex file for Benjamini-Hochberg
\par \indent When conducting multiple comparisons, it is important to have an 
idea of the quantity of Type I errors that may be prevalent in the analysis. 
In our analysis of voxel data, we decided that limiting/controlling the number 
of Type I errors is important to the process. The processes of limiting the 
number of Type I errors are called FDR-controlling procedures. In the grand 
scheme of things, FDR-controlling procedures give greater statistical power 
with the cost of more Type I errors that can fall through. 

\par Once we implemented the hypothesis function that would return t-test 
values and ``p-values'', we implemented the Benjamini-Hochberg procedure to 
control the proportion of rejected null hypotheses in the data. The Benjamini-
Hochberg procedure works by multiplying each of the ``p-values'' to a ratio of 
the number of tests and the chosen false discovery rate -- from these adjusted 
``p-values'', only the values that are less than the chosen false discovery rate 
will be chosen to be returned. This way, we are able to adjust the proportion of 
null hypotheses that will be rejected and the proportion that will return the 
desired proportion of significant tests. This will reduce the number of false 
positives returned in the data and extend greater statistical power in later 
analysis performed on the voxel dataset.
