% tex file for introduction
\par This paper, \textit{The Development and Generality of Self-Control} \cite{CohenSelfControl},  and its associated fMRI studies are concerned with the relationship between impaired and normal self control as well as the similarities and differences across the brain relating to self-control. The entire paper explores multiple studies (of multiple study types), but we will just focus on the third and final study, which compares four different types of self control among healthy adults to see if they are related to each other. Very little relationship was found between these behavioral tasks, in contrast to the vast majority of existing literature, which argues for a unified notion of self-control, which is why we have decided to narrow the data analysis approaches to just the BART study. The BART study primarily focused on the inflation/deflation of a balloon that could pop; the fMRI scans show blood flow to the brain in an attempt to reveal control over risk-taking behavior while subjects participate in this particular study.
\par The rest of this report will go further in detail on the processes that we have tried to mimic from the original analysis of the data, but here is a brief overview of the work we have accomplished so far.We have looked into spatial smoothing of the data on the voxels of the brain. After realizing that there are many factors that could contribute to noise in the data, we decided it would be beneficial to look into smoothing modules in Python that could handle n-dimensional datasets to clarify which voxels are more important than other voxels. After obtaining convolved time courses for each subject, we turned to fitting simple and multiple linear regression models to each subject. Since examining the behavior of blood flow in the voxels over time was of such great interest, we also considered modeling the behavior as a time series using an autoregressive integrated moving average ARIMA… (Quick Across Subjects, Hypotheis testing, PCA)
