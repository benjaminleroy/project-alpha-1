% tex file for introduction
\par \textit{The Development and Generality of Self-Control} \cite
{CohenSelfControl} and its associated fMRI studies are concerned with the 
relationship between impaired and normal self control, as well as 
similarities and differences across the brain relating to self-control. The 
paper in its entirety explores multiple studies (of multiple study types), 
but we will just focus on the third and final study, which compares four 
different types of self control among healthy adults to see if they are 
related to each other. Very little relationship was found between these 
behavioral tasks, in contrast to the vast majority of existing literature, 
which argues for a unified notion of self-control. So, we have decided to 
narrow our focus and data analysis approaches to just the Balloon Analogue 
Risk Task (BART) study, which purportedly measures control over risky 
behavior. fMRI scans from the study show blood flow to the brain, which may 
be relatable to control over risk-taking behavior during participation.

\par The rest of this report will detail the procedures from the original 
analysis of the data that we have tried to mimic. We experimented with 
different procedures to spatially smooth the voxels and the convolve and 
time-correct the time courses for each subject. After preprocessing the data, 
turned to fitting simple and multiple linear regression models to each subject, 
with the option of including study conditions. The resulting coefficients from 
the linear regression models can be used to perform t-tests to examine the 
significance of activity in different voxels. However, the validity of these 
tests is highly dependent on the validity of our model assumptions, such as the 
normality of our errors. Since we are performing a large quantity of tests, 
multiple comparison corrections are needed to control the false positive rate. 
We used k-means clustering to further identify regions of the brain across 
subjects with high activity and compared these results with the results of the 
original paper. 
