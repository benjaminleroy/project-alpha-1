% tex file for pca
\par \indent To perform Principal Components Analysis (PCA), we first 
considered the method outlined by the guide posted on the class website. 
Using matrix algebra, we were able to compute principal components by finding 
orthogonal projections (the shortest distance between points). After 
consulting with J.B. Poline in lecture, we took a new approach to PCA by 
implementing the Singular Value Decomposition (SVD) method for 4-dimensional 
data reshaped to 2-dimensional space. The SVD method is considerably more 
computationally efficient. 

\par PCA has the advantage of projecting the original data so that the 
predictors are independent. This can be used either to benefit interpretation 
or for dimension reduction. With our large number predictors (the time 
intervals), the original 2-dimensional matrix may be too large to interpret 
or study because there would be too many variables to look at. To analyze the 
data in a more meaningful form, PCA can be used to reduce the number of 
important variables to a few, interpretable linear combinations of our 
dataset. (Luckily, the components are obtained such that they decrease in 
variance explained.) Another potential use would be to examine the first few 
principal components of the residuals from linear regression to assess model 
fit and performance. 
