\documentclass[11pt]{article}

\usepackage[margin=0.75in]{geometry}
\usepackage{indentfirst}
\usepackage{graphicx}
\bibliographystyle{siam}

\title{Appendix: Outlier Removal}
\author{
  Chen, Kent\\
  \texttt{kentschen}
  \and
  Lee, Rachel\\
  \texttt{reychil}
  \and
  LeRoy, Benjamin\\
  \texttt{benjaminleroy}
  \and
  Liang, Jane\\
  \texttt{janewliang}
  \and
  Udagawa, Hiroto\\
  \texttt{hiroto-udagawa}
}

\begin{document}
\maketitle

We considered following the procedure implemented as part of Homework 2 to 
detect and remove outlier 3-d volumes from the 4-d image scans of each subject. 
The process involves finding the root mean squares (RMS) of each 3-d volume 
across time and then getting the difference values. When a given volume is very 
different from the preceding volume, this may indicate a potential outlier or 
the sign of an artificat. We used thresholds based on 1.5 \times IQR added to 
the 75th percentile and subtracted from the 25th percentile to create the 
cutoffs for the RMS difference outliers. We then extended the RMS difference 
outliers by labeling the volumes either side of the outlier RMS difference as 
being outliers. 

Each subject was considered independently, as considerable variation in 
measurement is expected between different subjects. However, we found that 
reductions in mean residual sum of squares from running simple linear regression 
before and after removing the extended RMS difference outliers were minimal. 
Visually speaking, we also did not observe the presence of egregiously different 
points. So, we opted to refrain from removing outliers, at least through the 
extended RMS difference method. 

\bibliography{project}

\end{document}
