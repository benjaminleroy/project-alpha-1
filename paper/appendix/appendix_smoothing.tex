% Smoothing appendix. 

\par \indent We used a Gaussian filter to smooth away noise in the brain 
images. At first, we played around with implementing a function similar 
to a mean filter, where each voxel would be the average of a certain radius 
of its neighbors. Ultimately, we decided to use a kernel with a Gaussian 
(bell-like) shape. While it is a linear type of smoothing, like the 
mean-smoothing approach, the Gaussian filter is a stronger choice than the 
mean-smoothing because it is not as affected by sharp spikes in the data. 

\par The property that makes the Gaussian filter a stronger and more 
reliable candidate for smoothing the voxel data is that it outputs a weighted 
average of the voxel and its neighbors. The more heavily weighted values are 
at the center of each of the neighborhood of voxels we examine. On the other 
hand, a regular mean filter would use a uniformly weighted average, which 
means that there is the risk of oversmoothing, along with additional 
complications of handling the voxels along the edges of the brain image data. 
Furthermore, a Gaussian filter is ideal for use in noisy voxel data because of 
its steady frequency response. By choosing an appropriate filter, we can 
gain more control of the range of spatial frequencies left after smoothing 
the data. Additionally, Gaussian filters are non-negative for all voxel data. 
Thus, the output of smoothing the voxel data with a Gaussian filter will still 
be a valid image. 

\par We decided to go with the module for a Gaussian filter for several reasons, 
the main one being that Gaussian filters can remove noise yet preserve the high 
frequency edges in the brain image data. Rather than using a self-implemented mean 
filter function that would cause issues with high-frequency edge cases as well as 
conglomerating data into thoughtless averages, we used a Gaussian filter which 
handles these situations better because the smooth, bell-shaped curve of the 
convolution does not have a sharp cutoff at edges. The Gaussian filter also 
distributes weighted averages across the voxels such that the smoothing keeps 
high-frequency data points into consideration for the end 
product.
