\documentclass[11pt]{article}

\usepackage[margin=0.75in]{geometry}
\usepackage{indentfirst}
\usepackage{graphicx}
\bibliographystyle{siam}

\title{Appendix: Smoothing}
\author{
  Chen, Kent\\
  \texttt{kentschen}
  \and
  Lee, Rachel\\
  \texttt{reychil}
  \and
  LeRoy, Benjamin\\
  \texttt{benjaminleroy}
  \and
  Liang, Jane\\
  \texttt{janewliang}
  \and
  Udagawa, Hiroto\\
  \texttt{hiroto-udagawa}
}

\begin{document}
\maketitle

\par \indent We used a Gaussian filter to smooth away noise in the brain 
images. At first, we played around with implementing a function similar 
to a mean filter, where each voxel would be the average of a certain radius 
of its neighbors. Ultimately, we decided to use a kernel with a Gaussian 
(bell-like) shape. 
\par The factor that makes a Gaussian filter a stronger and more 
reliable candidate to smooth the voxel data is that it outputs a weighted 
average of the voxel and its neighbors. The more heavily weighted values are 
at the center of each of the neighborhood of voxels we examine. On the other 
hand, a regular mean filter would use a uniformly weighted average, which means
 there is the risk of oversmoothingalong with additional complications of 
handling the voxels along the edges of the brain image data. Furthermore, a Gaussian filter is ideal for use in noisy voxel data 
because of its steady frequency response. By choosing an appropriate filter, 
we have more control of the range of spatial frequencies left after smoothing 
the data. Additionally, Gaussian filters are non-negative for all voxel data. 
Thus, smoothing voxel data with a Gaussian filter will guarantee non-
negativeness. Thus, the output of smoothing will still be a valid image.
\par In conclusion, we decided to go with the module for a Gaussian filter for 
several reasons, the main one being that Gaussian filters can remove noise and 
yet preserve the high frequency edges in the brain image data. Rather than use 
a self-implemented mean filter function that would cause issues with high-
frequency edge cases as well as conglomerating data into thoughtless averages, 
a Gaussian filter handles these situations better because the smooth bell-
shaped curve of the convolution does not have a sharp cutoff at edges, and the 
Gaussian filter also distributes weighted averages across the voxels such that 
the smoothing keeps high-frequency data points into consideration for the end 
product.
\bibliography{project}

\end{document}
