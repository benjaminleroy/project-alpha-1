% tex file for hypothesis

\par \indent We now have chosen our linear model and tested the normality
assumptions by analyzing the residuals from our model. Next, in order to
measure the strength of the association between the voxel time coures and the
HRF, we run a hypothesis test on the coefficients of the linear regression
model for each subject.

\par There is an individual linear model associated with each voxel in a
subject’s image (and a total of $64 \times 64 \times 34$ voxels per subject).
Thus we run a t-test on each voxel's $\beta_1$ coefficient, which is associated
with the HRF response. The null hypothesis for each test is that $ \beta_1=
0$, with the alternative hypothesis that $\beta_1 \neq 0$. Once we obtain
each t-statistic, we compare this value across voxels in two ways. First, we
simply compare the t-values with voxels within a subject. In this case, we
take into account the sign of the t-statistic in our analysis. Second, we
convert this t-statistic into a ``p-value'', in which case the sign of the
t-value will become irrelevant; we compare across voxels without taking
into account this sign. Later, we also run a multiple comparison test using a
Benjamini–Hochberg in order to find the voxels that are most significant.

\par Having implemented a method to compare the voxels within a single subject,
we next examine our results for the same voxels across subjects. Our initial
approach was to aggregate the t-statistic data between all subjects for each
voxel. This allows us to decrease the variability of the fit on each voxel and
detect a more clear signal.

\par In order to do this, we run the hypothesis test as stated above on all 24
subjects of the study. Then for each voxel, we take the average of the t-
statistics across the subjects. An issue with our data is that the presence of
empty space detected by the scanner is not directly part of the brain. To
account for this, we take the masked data of the brain and ``cut out'' the
parts of the images that were not relevant to our analysis. Ultimately, we are
left with a single 3-D image with each voxel representing the average
t-statistics across all subjects in the study. This image will later be used in
our clustering step in order to pinpoint the regions of the brain that have the
strongest relationship with the convolved time course.
