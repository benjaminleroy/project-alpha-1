% tex file for hypothesis

\par \indent We now have chosen our linear model and tested the normality
assumptions by analyzing the residuals from our model. Next, in order to
measure the strength of the association between the voxel time coures and the
HRF, we run a hypothesis test on the coefficients of the linear regression
model for each subject.

\par There is an individual linear model associated with each voxel in a
subject’s image (and a total of $64 \times 64 \times 34$ voxels per subject).
Thus we run a t-test on each voxel's $\beta_1$ coefficient, which is associated
with the HRF response. The null hypothesis for each test is that $ \beta_1=
0$, with the alternative hypothesis that $\beta_1 \neq 0$. Once we obtain
each t-statistic, we compare this value across voxels in two ways. First, we
simply compare the t-values with voxels within a subject. In this case, we
take into account the sign of the t-statistic in our analysis. Second, we
convert this t-statistic into a ``p-value'', in which case the sign of the
t-value will become irrelevant; we compare across voxels without taking
into account this sign. Later, we also run a multiple comparison test using a
Benjamini Hochberg in order to find the voxels that are most significant, and other clustering methods to analysis the $\hat{\beta}$ or t-values.

