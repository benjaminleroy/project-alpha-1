% tex file for hypothesis

\par \indent We now have chosen our linear model and tested its normality 
assumptions by analyzing the residuals from our model. Next, in order to 
measure the strength of the association between the voxel time courses and the 
HRF, we run a hypothesis test on the coefficients of the linear regression 
model for each subject. 

\par There is an individual linear model associated with each voxel in a 
subject’s image (and a total of $64 \times 64 \times 34$ voxels per subject). 
Thus we ran a t-test on each voxel's $\hat{\beta_1}$ coefficient, which is 
associated with the HRF response. The null hypothesis for each test is that 
$\hat{\beta_1}=0$, with the (two-sided) alternative hypothesis being 
$\beta_1 \neq 0$. Once we obtained each t-statistic, we compared this value 
across voxels in two ways. First, we simply compared the t-values with voxels 
within a subject. In this case, we took into account the sign of the 
t-statistic in our analysis. Second, we converted this t-statistic into a 
``p-value'' (assuming that the null distribution of the t-statistics is 
student's t), in which case the sign of the t-statistic becomes 
irrelevant. Here, we compared across voxels without taking into account this 
sign. We later performed a multiple comparisons correction using
Benjamini-Hochberg method in order to identify the voxels with the most 
significant activation. We also implemented clustering methods to analyze the 
$\hat{\beta}$ and t-statistics, taking into account the problem of multiple 
comparisons.

