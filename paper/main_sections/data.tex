% tex file for data
\par \indent The Balloon Analogue Risk Task (BART) measures risk-taking 
behavior by presenting participants with a computerized balloon. The 
participant can earn money incrementally by pumping up the balloon, but after 
an unknown threshold, the balloon will explode. At any time, the participant 
elect to cash out his or her earnings, but doing so eliminates the potential 
to gain additional money through pumps. If the balloon explodes, the 
participant loses all of the money for the trial. For this study, BART and 
fMRI data for 24 subjects was collected and deemed to be of good quality. 
The mean age of the subjects was 20.8, and ten of the subjects were female. 
Four behavioral variables were recorded for each subject: the average number 
of pumps for each balloon, the average amount of money earned across runs, the 
number of exploded balloons, and the number of trials. There were also three 
model conditions: events for inflating the balloon (excluding the very last 
inflation of each trial), the last inflation before an explosion, and the 
event of cashing out (the balloon explosion was not included as an event). Of 
interest for our work is the blood-oxygen-level dependent (BOLD) imaging data 
recorded for each subject during the course of task. Each subject's BOLD data 
was recorded as 64 by 64 image matrices in 34 slices, with a variable number 
of time points. 

\par Much of our analysis focuses on creating our own procedure for cleaning 
and preprocessing the raw BOLD data so that the true signal can be readily 
captured by our analyses. However, a cleaned version of the data was later 
made available by Ross Poldrack and the OpenfMRI project. The cleaned scans 
had receieved motion correction, high-pass filtering in time, and registration 
to the standard MNI anatomical template. Although we did not have the time to 
heavily investigate the pre-cleaned version of the data, it would have been 
interesting to compare the results from using our cleaning procedure versus 
the provided procedures.

\par It should also be noted that neither set of preprocessed data uses the 
exact same cleaning procedure as that used in \textit{The Development and 
Generality of Self-Control} \cite{CohenSelfControl}, which describes using 
various software and black-box methods that may not be available for our use. 
Additionally, while the paper is mostly concerned with comparing the areas of 
neural activity across different types of self control, we focused only on a 
single study on a single type of self control, for feasibility reasons. 
