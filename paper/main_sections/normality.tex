% tex file for normality
\par \indent The validity of our hypothesis tests of the estimated 
$\hat{\beta}$ values from linear regression is largely dependent on whether we 
can assume that the errors in our model(s) are independent and identically 
distributed from some normal distribution mean zero and constant variance. 
We focus here on checking the normality assumption. It is generally wise to 
use visualizations, such as residual vs. fitted values plots and 
quantile-quantile plots to inspect residuals for patterns and abnormalities. 
However, with the sheer quantity of data we are working with--- each of the 
24 subjects has 64 $\times$ 64 $\times$ 34 voxels that can each in turn be 
fitted to a model--- visual inspection is not practical. 

For this reason, we turned to using the Shapiro-Wilk test for normality, 
which tests the null hypothesis that the data in question is normally 
distributed. A Shapiro-Wilk test was performed for each set of residuals 
corresponding to a single voxel's time course. That is, each test used around 
200 observations, or the number of time points for that particular subject. 
200 observations is not an especially large sample size, and for this reason, 
we express some concern because normality tests have low power for small sample 
sizes. Shapiro-Wilk may incorrectly fail to reject the null hypothesis due to 
this bias \cite{ghasemi2012normality}. 
