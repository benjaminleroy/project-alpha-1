% tex file for future discussion
\par \indent The implementation of permutation tests, which have few 
assumptions and are easy to interpret (but computationally intensive), 
would have been useful for testing the significance of the estimated 
coefficients from linear modeling as an alternative to t-tests, which make 
several assumptsion about the data structure. Somewhat relatedly, 
additional tests and checks for model assumptions would also be valuable 
for assessing the appropriateness of our existing hypothesis testing. 

\par One major issue that we were unable to fully address is how to 
appropriately aggregate the results from individual subjects to make more 
general conclusions about activation regions in general, as opposed to the 
activation region of a particular subject. Much of this is due to the 
considerable variety in brain activity and shape observed between different 
subjects; there is really no such thing as an ``average'' subject. 

\par Additional future work could also be concerned with reproducing our 
approach for identifying actived regions on the pre-cleaned version of the 
study's data provided by OpenfMRI. Those results could then be compared
with the results from using our own pre-processing techniques. Since fMRI 
data is notoriously noisy, the different decisions made when cleaning the 
data to seperate the signal from the noise can greatly alter the results. 
So while ideally, both our pre-processed data and the data provided by 
OpenfMRI would identify the same active regions, we acknowledge that there 
is a reasonably high chance that this would not be the case. 
