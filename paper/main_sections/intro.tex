% tex file for introduction
\par \textit{The Development and Generality of Self-Control} \cite
{CohenSelfControl} and its associated fMRI studies are concerned with the
similarities and differences across the brain relating to different forms of 
self-control. The paper in its entirety explores multiple studies (of 
multiple study types), but we will only focus on the third study. The original 
study compares four different types of self control among healthy adults to 
see if the four are related to each other. Very little relationship was found 
between these different behavioral tasks --- in contrast to the vast majority 
of existing literature,  which argues for a unified notion of self-control. 
We have decided to narrow our focus and data analysis to just the Balloon 
Analogue Risk Task (BART) study, which purportedly measures control over 
risky behavior, for feasibility reasons. fMRI scans from the study show blood 
flow to the brain, which may be relatable to control of risk-taking behavior 
during participation.

\par We initially strove to reproduce the original analysis as faithfully as 
possible. Our lack of access to the packaged software used by Cohen to clean 
the data was one early limitation that led us to develop our own 
pre-processing pipeline. However, the more we progressed in our analysis, 
the more we found it difficult to justify pursuing all of the same analytical 
decisions made by the paper, since many of them involved unfamiliar or 
``black-box'' procedures. Furthermore, we identified several steps and 
assumptions in Cohen's analysis that we found difficult to theoretically 
justify. Thus, our final analysis actually deviates considerably from Cohen's 
approach. Our ultimate goal will be to compare the active regions identified 
by our analysis with those detected by the original paper. 

\par We experimented with different procedures to spatially smooth the voxels 
and to convolve and time-correct the time courses for each subject. After 
pre-processing the data, we built linear regression models for each subject's 
voxel time courses. Several design matrices for the models were explored, and 
model selection was used to identify optimal combinations of predictors, 
such as linear drift as well as Fourier series and principal components of the 
voxel time courses. The resulting estimated coefficients from the linear 
regression models can be used to perform t-tests to examine the significance 
of activity in different voxels. However, the validity of these tests is 
highly dependent on the validity of our model assumptions, such as the 
normality of our errors. Since we are performing a large quantity of tests, 
the Benjamini-Hochberg correction was used to control the false positive rate. 
Finally, we clustered the raw coefficients, t-statistics, and corresponding 
``p-values'' to identify regions of the brain with high activity. We then 
compared our results with those of the paper. 

