% tex file for discussion
\par \indent Cohen's paper pointed out sixteen regions of brain activation 
after the BART experiments on the subjects. Specifically, there were some 
activation in the dorsal lateral prefrontal cortex and through our analysis
on the subjects, some parts of activation was shown near the front of the 
brain. The prefrontal cortex covers the frontal lobe and functions as a 
storage of the brain's short term memory. Another region of the brain in which 
activation existed was the bilateral occipital pole and cortex near the rear 
of the brain. The occipital lobe is responsible for processing of visual 
information.

\par In the experiment, the participant relies on short-term memory for 
self-control. It seems that this action may possibly be activated by the 
prefrontal cortex. The short-term memory capabilities of the participant is 
essential as one needs to remember facts from previous trials such as how many 
pumps until the balloon popped or the decision on when to cash out. These are
important factors as the longer one holds out, the more risk associated. On the 
otherhand, it appears that the activation in the occipital lobe may be as a 
result of a side effect of the experiment. Though processing visual information 
is not a direct conclusion of the experiment, the brain can potentially be 
activated because of the nature of the activity of viewing the balloons, an 
indirect consequence of the BART experiment. 
 