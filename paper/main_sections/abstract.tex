% tex file for abstract
\par Self-control is an interesting field of behavioral research with broad 
implications for our day-to-day experiences. Being able to appropriately 
regulate and check our impulses and reactions to various everyday stimuli 
is necessary for maintaining health and high-functionality in society. 
Thus, relating a subject’s ability to control risky behavior to an area 
of the brain or focusing on understanding what connects self control to 
neurological activity has been the focus of many studies. One approach to 
capturing these neurological facets is to use functional magnetic resonance 
imaging (fMRI). Our goal is to identify active regions of the brain using 
fMRI data from a Balloon Analogue Risk Task (BART) study described in 
Cohen's \textit{The Development and Generality of Self-Control} 
\cite{CohenSelfControl}. When possible and logically sound, we will attempt 
to reproduce the data preprocessing and analysis outlined in the paper. 
However, many of our approaches deviate considerably from the methods 
used by Cohen, either out of necessity from our lack of pre-packaged 
software or when we were inspired explore other directions. Ultimately, we 
compare the active regions identified by our analysis with those detected 
by the original paper. 
