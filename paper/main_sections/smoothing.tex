% tex file for smoothing
\par \indent Due to the inherently random nature of human subjects and their 
movements, a certain level of smoothing must be performed on the spatial 
dataset. That way, the ``noisy'' data can be cast off from the data that 
actually represents significant changes in blood flow in the brain. By doing 
so, researchers and anyone else investigating the data will be able to 
distinguish between non-brain scans versus actual brain scans. Each voxel of 
the brain is represented by a measure of blood flow intensity, and so a 
series of steps must be taken so that the data is correctly convolved to most 
closely and accurately depict what was happening at a certain point in the 
brain at a certain time. After researching quite extensively, we decided to 
use smoothing involving a Gaussian kernel in order to smooth the three 
dimensional data. Originally, we were going to try and write a smoothing 
function from scratch, by implementing a rudimentary average-over-neighbors 
method. However, discussion with mentors lead us to the scipy module 
\texttt{ndimage.filters} that has a function to performs a Gaussian filter on 
n-dimensional data. This was exactly what we needed so rather than reinventing 
the wheel, we will be smoothing the data with this module. An in-depth 
discussion of the Gaussian filter can be found in the appendix\ref{app_smoothing}. 
