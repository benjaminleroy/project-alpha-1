% tex file for smoothing
\par \indent Due to the inherently random nature of human subjects and their 
movements, smoothing must be performed on the spatial dataset. That way, the 
``noisy'' data can be cast off from the data that actually represents 
significant changes in blood flow in the brain. Each voxel of the brain is 
represented by a measure of blood flow intensity, so a series of steps must 
be taken so that the data is correctly convolved to most closely and 
accurately depict what was happening at a certain point in the brain at a 
certain time. We decided to use smoothing via a Gaussian kernel in order to 
reduce noise in the three dimensional data. Originally, we were going to try 
and write a smoothing function from scratch, by implementing a rudimentary 
average-over-neighbors method. However, discussion with mentors lead us to the 
scipy module \texttt{ndimage.filters} that has a function to perform a
Gaussian filter on n-dimensional data. A more detailed discussion about our 
approach and the theory behind the smoothing of the hemodynamic response can 
be found in Appendix \ref{app_smoothing}.

