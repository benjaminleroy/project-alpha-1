% tex file for results

\noindent \textbf{Pre-Processing}

Smoothing performed well in reducing outliers and extreme values in the 
time-series per voxel. We used a value of $\sigma=1$ for the Gaussian 
smoothing. We performed HRF convolution and time correction as well with 
success.
\vspace{5mm}

\noindent \textbf{Model Selection}

For our linear regression model, we ended up choosing the design matrix model 
with a single HRF feature with all conditions, a linear drift feature, and 
three pairs of Fourier features. The dimension of our final feature matrix is 
time $\times$ 9. As noted previously, this selection of the model was also 
due to reduce problems with overfitting and collinearity with the HRF 
features. 
\vspace{5mm}

\noindent \textbf{Testing Normality Assumptions}

Based on the Shapiro-Wilk test for normality on the residuals with a p-value 
threshold of 0.05, we found that the normality assumptions of the linear 
models were violated roughly 35\% of the time. This suggests that we should be 
cautious of the validity of hypothesis testing and possibly explore 
alternatives for analyzing the coefficients.
\vspace{5mm}

\noindent \textbf{Clustering}
We used two perspectives to cluster, one based of multiple correction and 
another based on hierarchical clustering using Ward's method. Hierarchical 
clustering was computationally costly against the large number of voxels and 
was not feasible. We used 3 different approaches to multiple comparison 
correction. 

The most canonical approach to multiple correction we used was Benjamini-
Hochberg method, which utilizes p-statistics (with the underlying assumption 
of normality). Based on the fact that the normality assumptions did not hold 
for a large proportion of the voxels, we also utilized analyses that did not 
rely on such assumptions. Our other methods that mirrored our clustering 
approach to Benjamini-Hochberg, where we utilized the $\hat{\beta}$ and 
t-statistics corresponding to the HRF coefficient of the chosen linear model 
design matrix. We created active region clusters when the $\hat{\beta}$ and 
t-statistics for just absolute values of these statistics that were higher 
than 85\% of the rest.
\vspace{5mm}



\noindent \textbf{Identifying Active Regions}

We considered the results from each of the approaches for clustering with 
multiple comparison corrections. We tuned the false discovery rate cutoff for 
Benjamini-Hochberg and the quantile-based cutoffs for the t-statistic and 
$\hat{\beta}$ methods using subjects 1, 2, 3, 4, and 14. As alluded to 
previously, the BH approach was less interpretable and required per-subject 
rather than agglomerative tuning of parameters to obtain logical results. 
Thus, the BH results were largely dependent on the optimal choice of false 
discovery rate cutoff, which varied between subjects, and so the results from 
using the tuned parameter were not consistent. Thus, it was generally a less 
favorable approach than the t-statistic and $\hat{\beta}$ quantile-based 
clustering. 

These observations are visualized in Figures \ref{fig:clustert}, 
\ref{fig:clusterbeta}, and \ref{fig:clusterBH} (at the end of the paper, 
before the appendix), which compare the t-statistic, $\hat{\beta}$, and 
Benjamini-Hochberg clustering results, respectively, for a 
single subject (Subject 6). In the former two approaches, distinct active 
regions can be readily identified, whereas the Benjamini-Hochberg seems 
largely unable to pull out spatially grouped significant voxels. While some 
subjects had BH results that were reasonably interpretable, others subjects 
also experienced similarly scattered significant voxels. However, the other 
two methods were highly consistent with each other within the same subject. 
Small differences in the t-statistic and $\hat{\beta}$ results can largely be 
accounted by the presence of coefficient variances in the former. 

A region frequently identified with high HRF activity across subjects for the 
t-statistic and $\hat{\beta}$ approaches was the occipital area of the brain. 
Other areas with potentially high activity include a small region toward the 
center and front and some areas on the left edge. In Figures 
\ref{fig:clustersub3} and \ref{fig:clustersub11}, which show the 
quantile-based clustering results for the t-statistics, one can see the strong 
similarities in active regions for Subjects 3 and 11, particularly in the 
occipital area of the brain. There are some noticeable differences. The slices in which 
the occipital active regions are most prominent are shifted down for Subject 11 
compared to Subject 3. 

However, not all subjects behaved in this similar fasion, with the main active 
region being the occipital area of the brain. Among them were subjects with 
unusually-shaped brains, as well as other subjects like Subject 7 Figure 
\ref{fig:clustersub7}. Curiously, Subject 7's active regions seem almost 
backwards, with the highly active regions in the front of the brain and very 
little in the back. So the identified active regions are not always 
consistent or fully conclusive. 




