% tex file for hypothesis
\par \indent Our simple linear regression model was created not necessarily 
with prediction in mind, but rather to understand the relationship between 
the voxels in a given subject's brain and the convolved time course. In order 
to measure the strength of the association between these two measurements, we 
ran a hypothesis test on the coefficients of the simple linear regression 
model for each subject.

\par There is a distinct linear model associated with each voxel in a 
subject’s image (and a total of $64 \times 64 \times 34$ voxels per subject). 
So, we an a t-test on each voxel's non-intercept $\beta$ coefficient in our 
simple linear regression case. The null hypothesis for each test was that $
\beta = 0$, with the alternative hypothesis that $\beta \neq 0$. Once we had 
obtained each t-statistic, we compared this value across voxels in two ways. 
First, we simply compared this value across voxels within a subject. In this 
case, we took into account the sign of the t-statistic in our analysis. 
Second, we converted this t-statistic to a p-value, in which case the sign of 
the t-value will become irrelevant and we compared across voxels without 
taking into account this sign.  We also developed analogous functions to 
compute t-statistics for multiple linear regression. 

\par Having implemented a  method to compare the voxels within a single 
subject, we next examined our results for the same voxels across subjects. 
Our initial approach was to aggregate the t-statistic data between all 
subjects for each voxel. This allowed us to decrease the variability of the 
fit on each voxel and detect a more clear signal. 

\par In order to do this, we ran the hypothesis test as stated above on all 
24 subjects of the study. Then for each voxel, we took the average of t-
statistics across the subjects. An issue with our data was the presence of 
empty space detected by the scanner that is not directly part of the brain. 
In order to account for this, we took the masked data of the brain and ‘cut 
out’ the parts of the images that were not relevant. 
